\chapter{ $2025$年秋《矩阵理论》}

2025年秋研究生学位课程《矩阵理论》,共有$295$位同学选择此课程,包括计算机学院、海洋学院和机械学院的一部分学生.

%%%%%%%%%%%%%%%%%%%%%%%%%%%%%%%%%%%%%%%%%
\section{2025年10月23日:作业一}
% HOMEWORK - 作业一
\begin{tcolorbox}
\homework{1} 
设四阶实矩阵有如下两种展开形式:
\begin{equation*}
        \begin{pmatrix} 
        b_1^\top\\
        b_2^\top\\
        b_3^\top\\
        b_4^\top\\
        \end{pmatrix} 
\overset{\text{行展开}}{=:} 
    A(t) = 
        \begin{pmatrix} 
        t & 1 & 1 & -t\\
        1 & -t & t & 1 \\
        1 & t & -t & 1\\
        -t & 1 & 1 & t\\
        \end{pmatrix} 
      \overset{\text{列展开}}{:=} 
        \left( a_1,a_2,a_3,a_4\right),
\end{equation*}
其中列向量$b_1,b_2,b_3,b_4,a_1,a_2,a_3,a_4 \in \mathbb{R}^4$和$t \in \mathbb{R}$. 考虑矩阵$A(t)$的“行空间”和“列空间”分别是:
    \begin{equation*}
        \rm{Row(A)} = \rm{Span}\left\{b_1,b_2,b_3,b_4 \right\} ~ \text{和} ~
        \rm{Col(A)} = \rm{Span}\left\{a_1,a_2,a_3,a_4 \right\}.
    \end{equation*}
请回答以下问题(\textcolor{red}{ \large \bf 当前页不够,请在背面作答;提交纸质作业前,请拍照留存}): 
\begin{itemize}
\setlength{\itemsep}{0pt}
\setlength{\parsep}{0pt}
\setlength{\parskip}{0pt}
   \item[(Q1)] 请尝试求参数$t$的范围,使得线性子空间$\rm{Row(A)} $ 和 $ \rm{Col(A)}$的维数都是$4$;
   \item[(Q2)] 请尝试求参数$t$的范围,使得线性子空间的交$U = \rm{Row(A)} \cap \rm{Col(A)}$的维数为$2$,并将此线性子空间$U$的一组基扩充为$\mathbb{R}^{4}$的一组基;
   \item[(Q3)] 请尝试求参数$t$的范围,使得线性子空间的和$W = \rm{Row(A)} + \rm{Col(A)}$的维数为$2$, 并将此线性子空间$U$的一组基扩充为$W$的一组基.
 \end{itemize}
\end{tcolorbox}

\solve 首先,方程左右两侧同除以 $a$,得到
\[ x^2 + \frac bax + \frac ca = 0 \]
根据一次项来配方,按公式 $(x+A)^2=x^2+2Ax+A^2$ 配出常数项:
\[ x^2 + \frac bax + \left(\frac b{2a}\right)^2 + \frac ca - \left(\frac b{2a}\right)^2 = 0 \]
配方并移项得到
\[ \left(x + \frac b{2a}\right)^2 = \frac {b^2}{4a^2} - \frac ca \]
方程左右开方,得
\[ x + \frac b{2a} = \pm \sqrt{\frac {b^2}{4a^2} - \frac ca} \]
从而得到方程 \eqref{eq:quadratic} 之解为
\begin{equation}
  x = - \frac b{2a} \pm \sqrt{\frac {b^2}{4a^2} - \frac ca}
\end{equation}
该式即为一元二次方程的\textbf{通用求根公式}。


\analysis 在这一问题中,需要注意以下几点 \cite{texbook,latex}:
\begin{itemize}
  \item ……
  \item ……
  \item ……
\end{itemize}


%%%%%%%%%%%%%%%%%%%%%%%%%%%%%%%%%%%%%%%%%
\section{2025年10月30日:作业二}
% HOMEWORK - 作业二
\begin{tcolorbox}
\homework{2} 
如图 \ref{fig2:Rotation} 所示,点$\mathbf{v}$绕原点逆时针旋转 $\theta$, 得到点$\mathbf{v}'$. 假设点$\mathbf{v}$的坐标为$(x,y)^\top \in \mathbb{R}^2$ 和 点$\mathbf{v}'$的坐标为$(x',y')^\top \in \mathbb{R}^2$. 以上过程便构成了从$\mathbb{R}^2$绕原点到$\mathbb{R}^2$的一个{\bf 旋转变换} : $\mathscr{A}$,既,
\begin{equation}\label{Eq2:hw2}
    \begin{pmatrix}
        x'\\
        y'
    \end{pmatrix}
    = \mathscr{A} 
    \begin{pmatrix}
        x\\
        y
    \end{pmatrix}
\end{equation}

请回答以下问题(\textcolor{red}{ {\large \bf 当前页不够,请在背面作答;提交纸质作业前,请拍照留存} }): 
\begin{itemize}
\setlength{\itemsep}{0pt}
\setlength{\parsep}{0pt}
\setlength{\parskip}{0pt}
   \item[(Q1)] 请尝试确定 \eqref{Eq2:hw2} 中{\bf 旋转变换}$\mathscr{A}$的表示矩阵$A$;
   \item[(Q2)] 假设点$\mathbf{v}'$经过$x$和$y$方向分别平移$t_x$和$t_y$到点$\mathbf{v}'' = (x'',y'')^\top \in \mathbb{R}^2$,此过程确定了从$\mathbb{R}^2$到$\mathbb{R}^2$ {\bf 平移变换} : $\mathscr{B}$, 请尝试确定{\bf 平移变换} $\mathscr{B}$的表示矩阵$B$;
   \item[(Q3)] 在(Q1)和(Q2)的基础上,可以得到从点$v$到点$v''$的一个{\bf 线性变换} $\mathscr{C}$, 即
   \begin{equation}\label{Eq2:hw3}
    \begin{pmatrix}
        x''\\
        y''
    \end{pmatrix}
    = \mathscr{C} 
    \begin{pmatrix}
        x\\
        y
    \end{pmatrix}.
\end{equation}
    请尝试确定 \eqref{Eq2:hw3} 中{\bf 线性变换} $\mathscr{C}$的表示矩阵$C$,并在二维平面上给出相应的示意图.
 \end{itemize}
\end{tcolorbox}

\solve 首先,方程左右两侧同除以 $a$,得到
\[ x^2 + \frac bax + \frac ca = 0 \]
根据一次项来配方,按公式 $(x+A)^2=x^2+2Ax+A^2$ 配出常数项:
\[ x^2 + \frac bax + \left(\frac b{2a}\right)^2 + \frac ca - \left(\frac b{2a}\right)^2 = 0 \]
配方并移项得到
\[ \left(x + \frac b{2a}\right)^2 = \frac {b^2}{4a^2} - \frac ca \]
方程左右开方,得
\[ x + \frac b{2a} = \pm \sqrt{\frac {b^2}{4a^2} - \frac ca} \]
从而得到方程 \eqref{eq:quadratic} 之解为
\begin{equation}
  x = - \frac b{2a} \pm \sqrt{\frac {b^2}{4a^2} - \frac ca}
\end{equation}
该式即为一元二次方程的\textbf{通用求根公式}。


\analysis 在这一问题中,需要注意以下几点 \cite{texbook,latex}:
\begin{itemize}
  \item ……
  \item ……
  \item ……
\end{itemize}


%%%%%%%%%%%%%%%%%%%%%%%%%%%%%%%%%%%%%%%%%
\section{2025年10月30日:作业三}
% HOMEWORK - 作业三
\begin{tcolorbox}
\homework{3} 
如图 \ref{fig2:Rotation} 所示,点$\mathbf{v}$绕原点逆时针旋转 $\theta$, 得到点$\mathbf{v}'$. 假设点$\mathbf{v}$的坐标为$(x,y)^\top \in \mathbb{R}^2$ 和 点$\mathbf{v}'$的坐标为$(x',y')^\top \in \mathbb{R}^2$. 以上过程便构成了从$\mathbb{R}^2$绕原点到$\mathbb{R}^2$的一个{\bf 旋转变换} : $\mathscr{A}$,既,
\begin{equation}\label{Eq2:hw2}
    \begin{pmatrix}
        x'\\
        y'
    \end{pmatrix}
    = \mathscr{A} 
    \begin{pmatrix}
        x\\
        y
    \end{pmatrix}
\end{equation}

请回答以下问题(\textcolor{red}{ {\large \bf 当前页不够,请在背面作答;提交纸质作业前,请拍照留存} }): 
\begin{itemize}
\setlength{\itemsep}{0pt}
\setlength{\parsep}{0pt}
\setlength{\parskip}{0pt}
   \item[(Q1)] 请尝试确定 \eqref{Eq2:hw2} 中{\bf 旋转变换}$\mathscr{A}$的表示矩阵$A$;
   \item[(Q2)] 假设点$\mathbf{v}'$经过$x$和$y$方向分别平移$t_x$和$t_y$到点$\mathbf{v}'' = (x'',y'')^\top \in \mathbb{R}^2$,此过程确定了从$\mathbb{R}^2$到$\mathbb{R}^2$ {\bf 平移变换} : $\mathscr{B}$, 请尝试确定{\bf 平移变换} $\mathscr{B}$的表示矩阵$B$;
   \item[(Q3)] 在(Q1)和(Q2)的基础上,可以得到从点$v$到点$v''$的一个{\bf 线性变换} $\mathscr{C}$, 即
   \begin{equation}\label{Eq2:hw3}
    \begin{pmatrix}
        x''\\
        y''
    \end{pmatrix}
    = \mathscr{C} 
    \begin{pmatrix}
        x\\
        y
    \end{pmatrix}.
\end{equation}
    请尝试确定 \eqref{Eq2:hw3} 中{\bf 线性变换} $\mathscr{C}$的表示矩阵$C$,并在二维平面上给出相应的示意图.
 \end{itemize}
\end{tcolorbox}

\solve 首先,方程左右两侧同除以 $a$,得到
\[ x^2 + \frac bax + \frac ca = 0 \]
根据一次项来配方,按公式 $(x+A)^2=x^2+2Ax+A^2$ 配出常数项:
\[ x^2 + \frac bax + \left(\frac b{2a}\right)^2 + \frac ca - \left(\frac b{2a}\right)^2 = 0 \]
配方并移项得到
\[ \left(x + \frac b{2a}\right)^2 = \frac {b^2}{4a^2} - \frac ca \]
方程左右开方,得
\[ x + \frac b{2a} = \pm \sqrt{\frac {b^2}{4a^2} - \frac ca} \]
从而得到方程 \eqref{eq:quadratic} 之解为
\begin{equation}
  x = - \frac b{2a} \pm \sqrt{\frac {b^2}{4a^2} - \frac ca}
\end{equation}
该式即为一元二次方程的\textbf{通用求根公式}。


\analysis 在这一问题中,需要注意以下几点 \cite{texbook,latex}:
\begin{itemize}
  \item ……
  \item ……
  \item ……
\end{itemize}

 % 设置在代码块左部显示行号,用方框包围代码块,代码块显示为红色
% \begin{Verbatim}[numbers=left, frame=single, formatcom=\color{black}]
% using JuMP 

% using HiGHS
% wjModel = Model(HiGHS.Optimizer)  
 
% @variable(wjModel,x1 >= 0)
% @variable(wjModel,x2 >= 0) 

% @constraint(wjModel,3*x1 + 5*x2 <= 15) 
% @constraint(wjModel,3*x1 +   x2 <= 6)      

% @objective(wjModel,Max,2*x1 + 3*x2)

% print(wjModel)

% ## Solving...
% optimize!(wjModel)

% @show JuMP.value(x1)
% @show JuMP.value(x2)
% @show JuMP.objective_value(wjModel) 
% \end{Verbatim}

\begin{tcolorbox}
% breaklines:启用自动换行。很重要,不然代码会跑到页面外面去。
% breakanywhere:允许在任意位置换行。
% linenos:显示行号。
% tabsize=4:Tab 宽度
%fontsize=\small, linenos, breaklines, bgcolor=lightgray
\begin{minted}[fontsize=\small, breaklines,breakanywhere,tabsize=4]{julia}
function fib(n)
    if n <= 2
        return 1
    else
        return fib(n-1) + fib(n-2)
    end
end
println(fib(10))
\end{minted}
% 引用的时候需要
\captionof{listing}{"Hello, world" example}
\label{lst:hello_world}
% \vspace{1.5em}
\end{tcolorbox}



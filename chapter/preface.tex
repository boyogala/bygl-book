%!TEX encoding = UTF-8 Unicode
\chapter*{序}

 首先,非常感谢由 “\href{https://qyxf.github.io/}{钱院学辅} - \href{https://bjb.xjtu.edu.cn/}{西安交通大学钱学森书院辅导中心}”开发的、用于排版中等以上体量资料的{\bf \LaTeX{} 书籍模板:} 
 \begin{tcolorbox}
  \begin{description}
    \item [主要开发平台Gitee] \url{https://gitee.com/qyxf/qyxf-book}
    \item [开发平台GitHub] \url{https://github.com/qyxf/qyxf-book}
    \item [开发平台CTAN] \url{https://www.ctan.org/pkg/qyxf-book}
  \end{description}
\end{tcolorbox}
 

然而,为什么我要特别强调是给本科生使用的教材呢?结合我连续上这门课程《运筹与优化》的体验和感受,以及与学生的交流体会,和在课堂上的考察,我认真地思索并做了一些总结和归纳,大致地给出了几个方面的详细考虑(\textbf{个人观点,仅供参考}):
\begin{itemize}
	\item 繁复冗杂的知识堆砌;
	\item 偏重计算,忽视理论推演和证明过程,以致逻辑混乱;
	\item 实际问题偏少,没有启发性的问题代入以及趣味性; 
	\item 偏向毫无理由地模仿、调用和重复性耍题,不注重理论细节;
	\item 数学公式排版混乱,影响阅读以及学生自己推导和理解;
	\item 缺少编程思想和编程实践,导致解决实际问题的能力欠佳.
\end{itemize}  
 

为了更好地讲授《运筹与优化》,也为了更好地学习科学计算语言Julia, 为了把最优化理论及其算法应用到实际生产应用中,我开设了一个微信公众号《博优旮旯》,主要是利用优化理论和算法解决实际问题,请大家多多关注,多谢!
\begin{figure}[!htbp] 
\centering
\includegraphics[scale=0.7]{figure/boyogala-vx.pdf} % width=5.0in;
\caption{公众号《博优旮旯》}
\label{Chap1Sect1TuJieLP1} 
\end{figure}

为了方便大家,也为了方便我自己,更为了方便我教授的学生,我不仅在公众号《博优旮旯》里开通了一个“打编程”系列,里面几乎都是实际问题中所遇到的优化模型和相应的求解过程,而且决定把相应的数据、模型和代码公开 \href{https://github.com/boyogala}{https://github.com/boyogala} 上.
\begin{itemize}
    \item 运筹与优化:\href{https://github.com/boyogala/BeatORO}{https://github.com/boyogala/BeatORO};
    \item 打编程系列:\href{https://github.com/boyogala/BeatDBS}{https://github.com/boyogala/BeatDBS};
    \item 书友QQ群:$966589467$.
\end{itemize}
欢迎大家交流、使用和测试,请大家多多关注,多谢.
 

\begin{flushright}
王俊,马蔷
\par 2025年11月2日
\end{flushright}

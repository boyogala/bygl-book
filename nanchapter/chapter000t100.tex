
\chapter{数一数 001-100}

\section{001:为什么要了解数学的历史}


两个初识的人,还在互相隐晦地打探各自的过往历史,恨不得对其祖宗十八代都是摸排一次呢!

那,

数学历史是不是更应该了解和研究一下呢?
 
也许很多学习数学的学生,
或者是数学工作者,
都可能忽略了一个,
我个人认为的一个非常重要的问题:
我们为什么要研究数学的历史?
Image
数学史的主要研究对象是历史上的数学发现,调查它们的起源,或更广义地说,

数学史就是对过去的数学方法与数学符号的探究。

数学起源于人类早期的生产活动,为古中国六艺之一,亦被古希腊学者视为哲学之起点。

数学最早用于人们计数、天文、度量甚至是贸易的需要。

这些需要可以简单地被概括为

●数学对结构、空间以及时间的研究;

●对结构的研究是从数字开始的.

从我们称之为初等代数的——自然数和整数以及它们的算术关系式开始的,更深层次的研究是数论;

●对空间的研究则是从几何学开始的

从平面几何,即欧几里得几何和类似于三维空间(也适用于多或少维)的三角学。

后来产生了非欧几里得几何,在相对论中扮演着重要角色。

在公元前6世纪后,毕达哥拉斯将数学作为一门实证的学科进行研究,他创造了古希腊语单词μάθημα(mathema),意为“(被人们学习的)知识学问”。

希腊数学家在相当大的程度上改进了这些数学方法(特别引入了演绎推理和严谨的数学证明),并扩大了数学的主题。

中国数学做了早期贡献,包括引入了位值制系统。

如今大行于世的印度-阿拉伯数字系统和运算方法,很可能是在公元后1000年的印度逐渐演化.

从16世纪文艺复兴时期的意大利开始,算术、初等代数及三角学等初等数学已大体完备。

17世纪变数概念的产生使人们开始研究变化中的量与量的互相关系和图形间的互相变换。

随着自然科学和技术的进一步发展,为研究数学基础而产生的集合论和数理逻辑等也开始慢慢发展。

从古代到中世纪,数学发展的历史时期都伴随着数个世纪的停滞,但从16世纪以来,新的数学发展伴随新的科学发展,让数学不断加速大步前进,直至今日。

研究数学的历史,就好比研究人类的历史,文明的历史等等,只有知道从何处来,才能进一步地去寻找要去何处!

而对于每一个从事数学工作者来说,这也可以帮助我们从全局上,整体上来认识我们要研究的数学,以便能够进一步地认识她,也可为我们将来的道路指明一个方向。

然而,很多的学习数学的学生,从事数学的工作者,确实忘记了,他们很多人几乎从来都是不那么关心数学的历史的,在我看来,也未尝不是一大损失呀!

推荐数学史的书籍:

●卡尔·B.博耶,尤塔·C.梅兹巴赫《数学史(上下)》

●M.克莱因 《古今数学思想》(4册)

●克利福德皮寇弗《数学之书》

●杰克逊 《数学之旅 》

●迪厄多内 《当代数学:为了人类心智的荣耀》

●德夫林 《数学:新的黄金时代》

●卡斯蒂 《16-20世纪数学的五大指导理论》
